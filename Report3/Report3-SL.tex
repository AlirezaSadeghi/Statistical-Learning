\documentclass{article}
\usepackage{amsmath}
\usepackage{amsfonts}
\usepackage{amsthm}
\title{Statistical Machine Learning \\ Reading Assignment 3 Report}

\linespread{1.25}

\newtheorem{prop}{property}

\author{Alireza Sadeghi - Seyed Mohsen Shojaee}

\begin{document}

\maketitle

\section{Introduction}
The Bayesian approach to statistical problems has yielded very promising results. However in the context
of nonparametric problem is has been quite unsuccessful. One reason for this problem is the unsatisfied need for a manageable prior.
There are two essential properties for a prior distribution to work well in a nonparametric setting.
The Bayesian approach to statistical problems has yielded very promising results. However in the context
of nonparametric problem is has been quite unsuccessful. One reason for this problem is the unsatisfied need for a manageable prior.
There are two essential properties for a prior distribution to work well in a nonparametric setting.
\begin{itemize}
  \item The support of prior should be large to allow for many possible posteriors.
\item Posterior on observed data should be manageable analytically.
\end{itemize}
Dirichlet distribution introduced in \cite{fegusen} a prior
(probably the first one) for which the two properties above are simultaneously satisfied.

There are two alternate ways to define Dirichlet process. One by
describing the distribution on any finite subset and rely on
Kolmogorov's consistency theorem for the existence of the process.
The second way is by generalizing Dirichlet distribution to infinite dimension.

In this section, we first review the Dirichlet distribution. Then we provide two formal definitions and finally
give some insights about capabilities and shortcomings of Dirichlet Process i.e. the problems it can and can not model.

\subsection{Dirichlet Distribution}
A k-dimensional Dirichlet distribution is essentially a distribution over all probability measures on k possible outcomes.
For Example a 6-dimensional Dirichlet can be used to model the probability distribution of dices.
The formal definition is as follows.
Let $X_1, X_2, \ldots, \X_k$ be Gamma random variables with parameter $(\alpha_i, 1)$. A Dirichlet distribution with
parameter $(\alpha_1, \ldots, \alpha_n)$  is defined as distribution of $ Y = (Y_1, \ldots, Y_n)$, where
\begin{equation}
  Y_i = \frac{Z_i}{\sum_{i=1}^k Z_j}
\end{equation}
As can be seen in the above equation, $Y$ will lie on the $k-$simplex;
thus interpreted as a probability distribution over a sample space with $k$ members.

\subsection{Dirichlet Process}
We now turn to definition of Dirichlet process. Let $ \mathcal{X}$  be a measurable space
and $ \mathcal{X} $  a $\sigma-$field on it. The process is defined by specifying the joint distribution of any
Sequence $A_1, \ldots, A_n \in \mathcal{A}$.

To this end, it is sufficient to specify the distribution on every measurable portioning $B_1, \ldots, B_n$.
(We say portioning $(B_1, \ldots, B_n)$ to be measurable if $\forall i,j: B_i \in \mathcal{A}$ and $B_i \cap \B_j = \Phi$ and $\union_{j=1}^n B_j = \mathcal{X}$ )
From these distributions the distribution for arbitrary sequence $A_1, \ldots, A_n \in \mathcal{A}$ can be uniquely determined.
We are almost ready to give a formal definition for Dirichlet distribution, it only remains
the parameter of the process which should be a non-null finite measure on $\mathcal{X}$. We use $\alpha$ to denote the parameter.

We say a random process $Q$ is a Dirichlet process with parameter $\alpha$ if for every finite measurable portioning $B_1, \ldots, B_n$
the distribution of $(P(B_1), \ldots, P(B_n))$ is  Dirichlet with parameter $(\alpha(B_1), \ldots, \alpha(B_n))$.

An interesting property of Dirichlet distribution that can be considered a version of large support desired property is stated below.

Let $Q$ be a Dirichlet distribution, then for every fixed probability measure $T$ on $\mathcal{X}$ sequence of measurable sets
$A_1, \ldots, A_n \in \mathcal{A}$ and for every $\epsilon > 0$:
\begin{equation}
  \mathbb{P}(|P(A_i) - T(A_i) <\epsilon) > 0
\end{equation}

The following theorem ensures the posterior manageability for the Dirichlet process too,

\textbf{Therom 1} Let $P$ be a Dirichlet process on $(\mathcal{X},\mathcal{A})$ with parameter $\alpha $ and let $X_1,\ldots, X_n$
be a sample of size $n$ from $P$.
Then conditional distribution of $P$ given $ X_1, \ldots , X_n$, is a Dirichlet process with parameter $\alpha + \sum_{1}^n \delta_{X_i}$.




\section{Dirichlet Process}

\section{Sampling}

\section{Gaussian Process}

\end{document}
