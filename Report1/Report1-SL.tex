\documentclass{article}

\title{Statistical Machine Learning \\ Reading Assignment 1 Report}

\author{Alireza Sadeghi - Mohsen Shojaee}

\begin{document}

\maketitle
    
\section{A Taste of Real Analysis}

\subsection{Metric Space}

\subsubsection{Convergence \& Continuity}

\subsubsection{Topology of Metric Space}

\subsection{Topology: a More General Perspective}

\section{Theory of Probability}

\subsection{Introduction}
In many cases where statistics and statistical inference is an essential component of situation analysis, one encounters many discrete and continuous random variables and vectors and matrices. These are all special cases of a more general type of random quantity. The generalization of these notions to random quantities is through a notion of \textit{measure}.

\subsection{Measure Theory}
Measure, to be defined shortly is a way of assigning numerical values to the sizes of sets. Since it's used to give sizes to sets, it's domain is a collection of sets.

In order to define this "collection of sets" notion more thoroughly, we call a collection of sets that is closed under taking complements and finite unions, a \textbf{\textit{field}}.

A field, that is closed under taking countable unions is called a \textbf{\textit{$\sigma$-field}}.

A $\sigma$-field that is generated by the collection C of open subsets of a topological space is called a \textbf{\textit{Borel $\sigma$-field}}
\subsection{Measurable Functions}

Suppose \textit{S} is a set with a $\sigma$-field \textit{A} of subsets, and let \textit{T} be another set with a $\sigma$-field \textit{C} of subsets. Now consider a function $f : S \rightarrow T$. We say \textit{f} is \textit{measurable} if for every B $\in$ C, $f^{-1} \in A$.

If \textit{f} is measurable, one-to-one and onto, and $f^-1$ is also measurable, we say that \textit{f} is \textit{bimeasurable}, and also in another case, if the two sets \textit{S, T} are topological spaces with Borel $\sigma$-fields, a measurable function is \textit{Borel measurable}.


\subsection{Mathematical Probability}

\subsection{Random Variable}

\subsection{Conditioning}

\section{Stochastic Processes}

\subsection{Definition}

\subsection{Random Functions}

\end{document}
